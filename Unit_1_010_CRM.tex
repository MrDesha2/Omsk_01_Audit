%\input{head.tex}
\newpage

\subsection{Управление взаимоотношениями с клиентами}
\label{BP_CRM}

Поиском новых клиентов занимаются менеджеры отдела продаж.

Менеджеры ищут новых клиентов в различных открытых источниках, через обзвон потенциальных покупателей, через Интернет, систему 2Gis.
Менеджеры для поиска новых клиентов используют холодные звонки, выставки.

% Для продаж выделено 4 менеджера и 2 инженера по отгрузке.

CRM системы не выявлено.
Новых клиентов каждый менеджер ведет в своих файлах, блокнотах. Единой базы потенциальных покупателей не выявлено.

Новых клиентов бухгалтер заводит в справочник Контрагенты в системе 1С:УПП только при заключении договора.


В отделе продаж работает шесть человек: старший, ведущий, младший менеджер.
Выделено три группы менеджеров. Реально создано только две группы.

% Новые клиенты заносятся как лиды. МАП заносит карточку покупателя в модуле CRM системы 1С: УНФ. МАП ведет историю взаимоотношений с клиентами вручную записывая историю звонков, рассылок.
% В CRM созданы шаблоны сценариев работы с покупателем. Каждое событие работы с контрагентом является  элементом справочника с тем  шаблоном.
% При появлении потенциального клиента МАП определяет лицо принимающее решение со стороны организации. МАП запрашивает требования по изготовлению изделия и получает техническое задание (размеры и характеристики нового изделия). 

% В отделе продаж есть четкий регламент работы по новым заказчикам.
% Одна сделка представляет собой несколько изделий готовой продукции для одного клиента.



Опросного листа по клиенту не выявлено.
%
Общего списка потенциальных клиентов и холодных клиентов не выявлено.
Старший менеджер ведет таблицу в Excel по потенциальным клиентам (форма \ref{pic:d3}).
%
%
%
%%
%Учет взаимоотношений с клиентами ведется на ручном уровне. Менеджеры отдела маркетинга 
%занимаются поиском и привлечением новых покупателей. Выделяется пассивный поиск через сайт, рекламу и участие в мероприятиях, и активный поиск через адресную рассылку.
%Все контакты хранятся на сетевом ресурсе в сети ПРЕДПРИЯТИЯ, куда есть доступ большинству пользователей сети.
%При появлении нового покупателя менеджеры отдела маркетинга создают каталог в сетевом каталоге клиентов.
%Внутри каталога хранится информация по письмам с клиентами, договорам, спецификациям и тендерам.
%
%%
%Новых заказчиков ведет менеджер отдела продаж и закупок.
%После выполнения предварительных заказов заказчику ведущий специалист отдела продаж передает клиентов менеджерам отдела продаж.
%%
%%\begin{figure}
%%\begin{center}
%%\ifnum\pdfoutput=0
%%  \includegraphics[40,0][366,292]{Pics/TK1.png}
%%\else 
%%  \includegraphics[height=0.94\textheight, keepaspectratio]{Pics/CustomerList.jpg}
%%\fi
%%\end{center}
%%  \caption{Разделение потребителей гофротары в разрезе маркетологов}
%%  \label{pic:CustomerList}
%%\end{figure}
%%\clearpage
%%
%%
\begin{figure}
\begin{center}
 \includegraphics[height=0.5\textheight, angle=90, keepaspectratio]{Pics/d03.jpg}
\end{center}
 \caption{Список потенциальных покупателей}
 \label{pic:d3}
\end{figure}
\clearpage
%
%


\clearpage
\input {enddoc} 