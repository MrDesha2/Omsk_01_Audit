\newpage
\subsection{Изготовление образцов продукции}
\label{bp:pattern}

% Вырезание образцов на плоттере. Оператору на плоттер приходит задание с чертежами в AUTOCADE по электронной почте. Заготовка попадает в план на гофроагрегат. Оператор плоттера смотрит план и ориентируется на графу «номенклатура» и «упаковка». Если они не заполнены, значит это заготовки на плоттер (ф 86). Далее он переносит чертеж из AUTOCADE в свою программу Impact (ф 89) и вырезает образцы. Далее, при необходимости склеивает их, упаковывает, прикрепляет ярлык (ф 88) и передает на склад готовой продукции, заполнив отчет (ф 90), который служит подтверждением передачи на склад. После заполняет отчет на сервере (ф 87) по которому ориентируются о выполнении заказчики образцов.




После поиска потенциального потребителя менеджер запускает процедуру изготовления образцов при необходимости.
%Образцы изготавливаются только на основании предоставленного заказчиком макета.

Менеджер создает служебную записку (нет утвержденной формы) на изготовление образцов и с макетом от клиента передает в технологический отдел. 

 менеджер создаёт служебную записку вручную, подписывают у генерального директора, посылает директору по производству. История по изготовлению образцов не ведётся. История по изготовлению образцов ведется только в переписке с клиентом. 
 Образец сложной высечки делают только один экземпляр, либо на плоттер через предприятие ООО ''Алтайтара'' со сроком изготовления 2 недели.
 
 Плоттера на предприятии нет.
 
% При поступлении заказа на образцы конструктор разрабатывает  чертеж изделия. Затем конструктор вырезает образец на плоттере.
% Для образцов конструктор берет заготовки с производства подходящий по размерам, марке, цвету и профилю. В редких случаях заводят заказ на образцы.
% %Конструктор на основании полученной служебной записки и макета вырезает на плоттере образец. 
% На плоттере можно вырезать партию до 50 штук. При необходимости большего объема менеджер создает внутреннюю заявку для производства заготовок на гофроагрегате под образцы.
% Менеджер самостоятельно опрашивает конструктора о готовности образцов.


Образцы производятся за счет Предприятия.

%Менеджер при обращении покупателя создаёт служебную записку на изготовление образцов (форма \ref{pic:pic_d1}). Образцы продукции изготавливает вручную специалист в конструкторском отделе.  %Максимальное количество образцов изготавливается до 10 штук. Менеджер оставляет копию служебной записки у себя для контроля производства образцов. 
%Журнал учёта образцов не ведётся. 
%Вырезается на один образец больше. Дополнительный образец хранится в отделе штампов до принятия окончательного решения заказчиком. Готовые образцы передаются менеджерам с ярлыком. 


% \begin{figure}
% \begin{center}
%   \includegraphics[height=0.94\textheight, width=0.9\textwidth, keepaspectratio]{Pics/pic_d1.jpg}
% \end{center}
%   \caption{Заявка на изготовление образцов}
%   \label{pic:pic_d1}
% \end{figure}
% \clearpage

\ifx \notincludehead\undefined
\normalsize
\end{document}
\fi