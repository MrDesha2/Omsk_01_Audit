% \newpage
\subsection{Претензии контрагентов}
\label{BP_CLaim} 

% Регистрации претензии контрагентов не выявлено. 
% Претензии поступают редко. Менеджеры принимают претензии, но при этом  не ведут их регистрацию.
Претензии от заказчиков поступают в среднем 2-3 раза в неделю. 
Претензии регистрирует менеджеры в бумажном журнале.

Основные проблемы связаны с качеством нанесения печати, геометрии изделия, коробления, нарушения целостности упаковки.
Претензии поступают менеджерам по почте или мессенджерам. 
При поступлении официального письма от клиента менеджер делает рассылку по всем отделам, кому адресована претензия. 

% (рис. \ref{pic:d29}).

После разбора причин выявления брака менеджер пишет ответ на официальное письмо или договаривается с заказчиком о решении вопроса.

% В отдел ОТК претензия поступает по электронной почте от менеджера. Расследование проводят служба ОТК, находят рапорт, на эту дату и ответственного. Заметки делают прям на актеа(рис. \ref{pic:a21}).

% Претензии по сырью составляется службой ОТК (рис. \ref{pic:a22}) и передают Генеральному директору для урегулирования с поставщиками сырья.

% \begin{figure}
% \begin{center}
%   \includegraphics[height=0.6\textheight, angle =-90, keepaspectratio]{Pics/d29.jpg}
% \end{center}
%   \caption{Список поступивших претензий}
%   \label{pic:d29}
% \end{figure}

% \begin{figure}
% \begin{center}
%   \includegraphics[height=0.6\textheight, angle =0, keepaspectratio]{Pics/a21.jpg}
% \end{center}
%   \caption{Пример претензии заказчика}
%   \label{pic:a21}
% \end{figure}

% \begin{figure}
% \begin{center}
%   \includegraphics[height=0.6\textheight, angle =0, keepaspectratio]{Pics/a22.jpg}
% \end{center}
%   \caption{Пример претензии поставщику}
%   \label{pic:a22}
% \end{figure}

\clearpage
\ifx \notincludehead\undefined
\normalsize
\end{document}
\fi